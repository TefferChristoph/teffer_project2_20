\documentclass[12pt,a4paper]{scrartcl}
\usepackage[utf8]{inputenc}

\usepackage{graphicx}
\usepackage[ngerman]{babel}
\usepackage[none]{hyphenat}
\usepackage{listings}
\usepackage{blindtext}
\usepackage{float}


\begin{document}
\begin{titlepage}
	\centering
	{\scshape\LARGE HTL Wiener Neustadt \par}
	\vspace{1cm}
	{\scshape\Large NVS Projekt\par}
	\vspace{1.5cm}
	{\huge\bfseries Section Controll \#20\par}
	\vspace{2cm}
	{\Large\itshape Teffer Christoph (20), 5CHIF\par}
	\vfill
	\vfill

% Bottom of the page
	{2 Semester}
\end{titlepage}

\tableofcontents
\newpage

\section{Aufgabenstellung}
\label{sec:Aufgabenstellung}

Aufgabe ist es ein netzwerk-basierte Lösung für eine Section Controll, 
welche die Zeit von Autos, in einer gewissen Distanz misst. Mithilfe dieser 
Zeit und der bekannten Distanz kann man sich nun die Geschwindigkeit des Autos
ausrechnen. Bei einer Überschreitung der erlaubten Geschwindigkeit wird die Polizei
alamiert, die in drei Minuten an der Poistion an der sich das zu schnelle Auto eintreffen wird.

\section{Idee}
\label{sec:Idee}

Die Zeit, die ein Auto für das passieren der Messtelle braucht, wird zufällig, 
in einem vom Benutzer angegebenen Bereichs ermittelt.
Die Messtelle ist der Client. Dem Client gibt man über die Kommandozeile
die Ditanz, in Meter, in der gemessen wird, die minimale Zeit und die Maximale Zeit.
Führt man das Client-Programm aus, werden nun unendlich viele Autos simuliert.
Sollte ein Auto zu schnell sein wird den Server die Geschwindigkeit des Autos gesendet.
Am Server, auf der Polizeistation, wird dann der Weg den das Auto innerhalb der nächsten drei Minuten zurücklegen wird
ausgerechnet. 

\section{Implementierung}
\label{sec:Implementierung}

Das Projekt besteht aus zwei CPP Dateien, eine für den Client und die 
Andere für den Server. Die Kommunikation läuft synchron ab. Der Client sendet
bei einer zu hohen Geschwindigkeit den Server die Geschwindigkeit und der Server rechnet sich
den Weg, welchen das Autos in den nächsten drei Minuten zurücklegen wird aus. Der Server gibt die Ergebnise nach
drei Minuten auf der Konsole aus.
Einegebaute Bibliotheken sind CLI11 (Parsen der Kommandozeilenparameter) und ASIO (Kommunikation).

\section{Kommunikation}
\label{sec:Kommunikation}

Die Kommunikation wurde mit Hilfe der Bibliothek ASIO realisiert. Server und Client 
kommunizieren mit einem tcp::iostream. Die Kommunikation läuft synchron ab. Gesendet werden Strings, die Geschwindigkeit.


\subsection{Client}
\label{subsec:Client}

Bis das Programm nicht beendet wird, werden alle drei Sekunden Autos simuliert. Diese Autos fahren in einer zufällig ermittelten Zeit
eine gewisse Distanzn. Ist ein Auto zu schnell wird den Servre die Geschwindigkeit des Autos übermittelt. Die Geschwindigkeit kann man sich 
mit der Formel Weg/Zeit ausrechnen. Allerdings sind das Meter pro Sekunde. Multiplieziert man das Ergebnis mit 3.6 erhählt man die Geschwindigkeit
in km/h. 
Ist ein Auto im erlaubten Bereich wird den Server nichts gesendet.
Die Zeit und die Geschwindigkeit des Autos werden auf der Konsole ausgegeben
Sollte keine Verbindung zum Server vorhanden sein, wird das Programm mit einer Fehlermeldung beendet.
\subsection{Server}
\label{subsec:Server}

Der Server nimmt die Client Verbindung an. Die Einsatzmeldung der Polizei läuft als eigener Thread ab. Und beendet sich wenn er keine Signale mehr bekommt.
In einer Funktion die der Thread dabei ausführt passiert in den ersten drei Minuten gar nichts. Dann wird ein lockguard mit einem mutex erstellt. Zum Schluss wird ein Vector,
indem ein Tupel welches die Geschwindigkeit und den benötigten Weg enthält, durchgegangen um sich die Werte zu holen. Diese Werte werden dann ausgegeben.

Solange noch Signale vom Client kommen wird der Thread vom Objekt Thread getrennt, damit man sich den Weg den die Polizei benötigen wird
ausrechnen kann. Der ausgerechnete Weg wird zusammen mit der Geschwindigkeit hinten in den Vector gepusht.








\end{document}